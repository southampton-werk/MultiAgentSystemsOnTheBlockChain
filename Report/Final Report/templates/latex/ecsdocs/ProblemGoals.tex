%% ----------------------------------------------------------------
%% Problem.tex
%% ---------------------------------------------------------------- 
\chapter{Project Problem and Goals} \label{Chapter: Project Problem and Goals}
\section{Problem}
Most university clubs collect registration fees to finance their activities for the year. The current solution for collecting the fees, is either with cash or online. The problem for both systems is that the money is collected by one entity, and cash has the added negative that the payment could be repudiated. At the annual general meeting  (AGM) , members vote for representatives and for managers. A problem arises when the AGM is held before the start of semester one, and this means that many members and first year students, who are a large proportion of the club members, are unable to vote. \\
After representatives and managers are chosen, they will decide on several items where budget can be spent and how the money is allocated to those items. The disadvantage with this, is that deciding where the money should go, can take a long time. Another problem with the negotiation phase is that quiet individuals may be bullied into making concessions by someone more confident. Another negative is that currently many systems use the budget negotiation algorithm; sum all the representative’s budgets and divide by the number of representatives. This is very easy to manipulate by overexaggerating one expenditure, in order to make that expenditure larger in the final budget. Lastly, the final budget may not be known by the entirety of the club, making it unaccountable and opaque. \\
Also, the current systems run on private servers and databases, and therefore users cannot trust that the owners of the systems are not manipulating the data. Addionaly the servers and databases require an upfront cost.
\section{Goals}
For the collection of the fees, I want my solution to have certain characteristics. First off, all registrations are non-repudiable. This means that there is a method for confirming that the user has paid the fee. Also, we can check the authenticity of a registration. This means that a user can identify himself as the one who paid the registration. Another feature is transparency; everyone should be able to see how much money has been collected from fees. Lastly, until the budget has been decided, no one should be able to access the money. \\
For the voting, the features are as follows; Firstly, all members who have paid their registration fees should have equal voting power and the ability to apply as a representative. Another feature is the ability for a user to submit a full ranking of representatives, to allow him a larger ability to customize his vote. We also want to give each representatives a weight, to take into account the fact that some of them will receive more votes than their counterparts and therefore should have a larger mandate. Lastly, the system should discourage tactical voting, as we want people to submit their votes that represent their opinion, and it not swayed by how other people are voting. \\
For the negotiation and distribution of the budget, the beneficial characteristics are thus; representatives can submit sources of expenditures and their preferred budget. During this pre-negotiation phase, all messages sent to each other are recorded and non-repudiable, and this should increase accountability, because representatives will be held to the claims they made. The system will then run an automated negotiation algorithm, which solves the problem of representatives’ personality affecting the outcome. The negotiation algorithm should be deterministic and therefore other users can check the integrity of the budget. The algorithm should make it difficult to manipulate the final budget. Lastly for the distribution of the budget, after the negotiation, the money should be sent automatically to the managers of the expenditures.  







