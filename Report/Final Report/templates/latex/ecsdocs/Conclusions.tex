%% ----------------------------------------------------------------
%% Conclusions.tex
%% ---------------------------------------------------------------- 
\chapter{Conclusions} \label{Chapter: Conclusions}
The System produced solves the current problem with clubs, where all the money is collected and spent by one or a few people. This problem is made worse at university clubs, where is its likely that a first year student has never seen the club before and therefore has no trust in the club or its leadership.  \\
My solution is a smart contract which is run on the Ethereum network. Users can register, vote and decide the budget by sending transactions to the contract. Users do not have to trust the integrity of the smart contract, due to the Ethereum miners who validate every transaction. \\
One problem added, was that due to the complete transparency, malicious users could use the information of other users' votes and budget to manipulate the result of a election or final budget. I had to choose between simple algorithms that could be easily manipulated and cheaply run or algorithms that were hard to manipulate but cost more to run. I choose the second option, but tried to reduce the gas price as much as possible. \\
My system can scale well for users registering and voting. Although number of representatives and candidates must be kept low or the gas price starts to become too high. It can also be mitigated by increasing the coalition size factor, but this makes the negotiation pointless. However, for the last couple of months the gas to USD value has been trending down and so in the future may allow for more representatives. \\
Throughout the year other turing complete Cryptocurrencies were created, such as EOS, that claimed to be able to handle a much larger number of transactions then Ethereum. This meant that the cost to run programs on their network was much cheaper. The problem is that distributed systems are easy to scale when there are so few users. Ethereum used to be as cheap as EOS. Another problem is that EOS has a lot less computing power securing the network with mining, and so a 50\% is a lot easier. Lastly Ethereum already has a large number of distributed applications on the network, and will likely stay the main blockchain for distributed applications due to the network effect. 
I have also written a simple html and javascript interface for my smart contract. This allows users to interact with the smart contract without a command line. 
